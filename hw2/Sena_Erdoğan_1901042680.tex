\documentclass[a4 paper]{article}
\usepackage[inner=2.0cm,outer=2.0cm,top=2.5cm,bottom=2.5cm]{geometry}
\usepackage{setspace}
\usepackage[rgb]{xcolor}
\usepackage{verbatim}
\usepackage{subcaption}
\usepackage{amsgen,amsmath,amstext,amsbsy,amsopn,tikz,amssymb}
\usepackage{fancyhdr}
\usepackage[colorlinks=true, urlcolor=blue,  linkcolor=blue, citecolor=blue]{hyperref}
\usepackage[colorinlistoftodos]{todonotes}
\usepackage{rotating}
\usepackage{booktabs}
\newcommand{\ra}[1]{\renewcommand{\arraystretch}{#1}}

\newtheorem{thm}{Theorem}[section]
\newtheorem{prop}[thm]{Proposition}
\newtheorem{lem}[thm]{Lemma}
\newtheorem{cor}[thm]{Corollary}
\newtheorem{defn}[thm]{Definition}
\newtheorem{rem}[thm]{Remark}
\numberwithin{equation}{section}

\newcommand{\homework}[6]{
   \pagestyle{myheadings}
   \thispagestyle{plain}
   \newpage
   \setcounter{page}{1}
   \noindent
   \begin{center}
   \framebox{
      \vbox{\vspace{2mm}
    \hbox to 6.28in { {\bf CSE 211:~Discrete Mathematics \hfill {\small (#2)}} }
       \vspace{6mm}
       \hbox to 6.28in { {\Large \hfill #1  \hfill} }
       \vspace{6mm}
       \hbox to 6.28in { {\it Instructor: {\rm #3} \hfill  {\rm #5} \hfill  {\rm #6}} \hfill}
       \hbox to 6.28in { {\it Assistant: #4  \hfill #6}}
      \vspace{2mm}}
   }
   \end{center}
   \markboth{#5 -- #1}{#5 -- #1}
   \vspace*{4mm}
}

\newcommand{\problem}[2]{~\\\fbox{\textbf{Problem #1}}\hfill (#2 points)\newline\newline}
\newcommand{\subproblem}[1]{~\newline\textbf{(#1)}}
\newcommand{\D}{\mathcal{D}}
\newcommand{\Hy}{\mathcal{H}}
\newcommand{\VS}{\textrm{VS}}
\newcommand{\solution}{~\newline\textbf{\textit{(Solution)}} }

\newcommand{\bbF}{\mathbb{F}}
\newcommand{\bbX}{\mathbb{X}}
\newcommand{\bI}{\mathbf{I}}
\newcommand{\bX}{\mathbf{X}}
\newcommand{\bY}{\mathbf{Y}}
\newcommand{\bepsilon}{\boldsymbol{\epsilon}}
\newcommand{\balpha}{\boldsymbol{\alpha}}
\newcommand{\bbeta}{\boldsymbol{\beta}}
\newcommand{\0}{\mathbf{0}}


\begin{document}
\homework{Homework \#2}{Due: 07/12/20}{Dr. Zafeirakis Zafeirakopoulos}{Gizem S\"ung\"u}{}{}
\textbf{Course Policy}: Read all the instructions below carefully before you start working on the assignment, and before you make a submission.
\begin{itemize}
\item It is not a group homework. Do not share your answers to anyone in any circumstance. Any cheating means at least -100 for both sides. 
\item Do not take any information from Internet.
\item No late homework will be accepted. 
\item For any questions about the homework, send an email to gizemsungu@gtu.edu.tr
\item The homeworks (both latex and pdf files in a zip file) will be
submitted into the course page of Moodle.
\item The latex, pdf and zip files of the homeworks should be saved as
"Name\_Surname\_StudentId".$\{$tex, pdf, zip$\}$.
\item If the answers of the homeworks have only calculations without any formula or any explanation -when needed- will get zero.
\item Writing the homeworks on Latex is strongly suggested. However, hand-written paper is still accepted $\textbf{IFF}$ hand writing of the student is clear and understandable to read, and the paper is well-organized. Otherwise, the assistant cannot grade the student's homework.
\end{itemize}

\problem{1: Relations}{15}
Draw the Hasse diagram for the “greater than or equal to” relation on $\{$0, 1, 2, 3, 4, 5$\}$. Show each step to build the diagram. In order to draw the diagram, you can choose one of the following options:

\begin{itemize}
	\item You can use an online drawing tool such as https://app.diagrams.net/.
	\item You can draw the diagram by hand, take a picture of it to put on Latex as a figure.
	\item Word Office, Libre Office are also options to use them as drawing tools. 
\end{itemize}
\solution

To draw the Hasse diagram, the given set should be a poset so it needs to be reflexive, anti-symmetric and transitive. Every element of this set is equal to itself, satisfying the condition of a being greater than or equal to b for (a,b), making the given set reflexive. For anti-symmetry, (a,b) and (b,a) can coexist only if a and b are equal. If (a,b) is valid, a is greater than or equal to b. If that is the case (b,a) cannot be valid unless a and be are equal, so the given set is anti-symmetric. For every (a,b) and (b,c) in this set there exists a (a,c), making the set transitive. This given set is a poset.

\newpage

\begin{figure}[h!]
  \centering
  \begin{subfigure}[b]{0.3\linewidth}
    \includegraphics[width=\linewidth,height=\linewidth]{hasse-1.png}
    \caption{Figure 1}
  \end{subfigure}
  \begin{subfigure}[b]{0.3\linewidth}
     \includegraphics[width=\linewidth,height=\linewidth]{hasse-2.png}
    \caption{Figure 2}
  \end{subfigure}
  \begin{subfigure}[b]{0.3\linewidth}
    \hbox{\hspace{15ex}\includegraphics[width=0.2\linewidth,height=\linewidth]{hasse-3.png}}
    \caption{Figure 3}
  \end{subfigure}
\end{figure}

(5,5), (5,4), (5,3), (5,2), (5,1), (5,0), (4,4), (4,3), (4,2), (4,1), (4,0), (3,3), (3,2), (3,1), (3,0), (2,2), (2,1), (2,0), (1,1), (1,0) and (0,0) needs to be illustrated on the Hasse diagram. The elements are written vertically going upwards, starting from 5 and going to 0 since the arrows should look up. Then the arrows are drawn going from a to b in an (a,b) relation as seen in Figure 1. After all relations are covered, arrows for reflexivity and transitivity should be removed. Reflexivity is represented by loops so all the loops in the diagram are removed. Transitivity states that if the relation has (a,b) and (b,c), then there should be (a,c) in the relation too and that (a,c) should be removed from the Hasse diagram, which is shown in Figure 2. After all the steps are done, the arrowheads should be removed, there should only be a line between two elements. Figure 3 shows the Hasse diagram for the given set.

\problem{2: Relations}{15}
Answer these questions for the poset ($\{\{$1$\}$, $\{$2$\}$, $\{$4$\}$, $\{$1, 2$\}$, $\{$1, 4$\}$, $\{$2, 4$\}$, $\{$3, 4$\}$, $\{$1, 3, 4$\}$, $\{$2, 3, 4$\}\}$, $\subseteq$).
\subproblem{a} Find the maximal elements.
\solution

If there is no element upper than a in the Hasse diagram, a is a maximal element. In this case, the maximal elements should be the elements with the greatest cardinality since they can't be the subset of another subset. The maximal elements are {1,3,4}, {2,3,4}.

\subproblem{b} Find the minimal elements.
\solution

If there is no element lower than a in the Hasse diagram, a is a minimal element. In this case, the minimal elements should be the elements with the lowest cardinality since they will always be the subset of another subset. The minimal elements are {1}, {2}, {4}.

\subproblem{c} Is there a greatest element?
\solution

A greatest element should be the only maximal element for a poset. This poset has 2 maximal elements therefore there is not a greatest element for this case.

\subproblem{d} Find all upper bounds of $\{\{$2$\}$, $\{$4$\}\}$.
\solution

Upper bounds should be greater than or equal to the given element so every element of the subset should be checked for if they are greater than or equal to both {2} and {4}. If an element satisfies the condition for {2} but not for {4}, then that element cannot be considered as an upper bound. {2,4} and {2,3,4} satisfy the condition.

\subproblem{e} Find the least upper bound of $\{\{$2$\}$, $\{$4$\}\}$, if it exists.
\solution

Least upper bound will be the element with the least cardinality out of all the upper bounds. Comparing {2,4} and {2,3,4}, {2,4} would be the least upper bound since it's cardinality is 2 and {2,3,4}'s cardinality is 3.

\subproblem{f} Find all lower bounds of $\{\{$1, 3, 4$\}$, $\{$2, 3, 4$\}\}$.
\solution

Lower bounds should be less than or equal to the given element so every element of the subset should be checked for if they are less than or equal to both {1,3,4} and {2,3,4}. If an element satisfies the condition for {1,3,4} but not for {2,3,4}, then that element cannot be considered as a lower bound. {1}, {4}, {1,4} and {3,4} satisfy the condition.

\subproblem{h} Find the greatest lower bound of $\{\{$1, 3, 4$\}$, $\{$2, 3, 4$\}\}$,
if it exists.
\solution

Greatest lower bound will be the element with the greatest cardinality out of all the lower bounds. Comparing {1}, {4}, {1,4} and {3,4}, There wouldn't be a greatest lower bound since {1,4} and {3,4} have the same cardinality and a greatest lower bound should be unique. There is no greatest lower bound.

\problem{3: Relations}{70}
Remember that a relation $R$ on a set $A$ can have the properties reflexive, symmetric, anti-symmetric and transitive.

\begin{itemize}
	\item $\textbf{Reflexive: }$ R is reflexive if (a, a) $\in$ $R$, $\forall$ a $\in$ A.
	\item $\textbf{Symmetric: }$ R is symmetric if (b, a) $\in$ R whenever (a, b) $\in$ R, $\forall$ a, b $\in$ A.
	\item $\textbf{Anti-symmetric: }$ R is antisymmetric if $\forall$ a, b $\in$ A, (a, b) $\in$ R and (b, a) $\in$ R implies that a = b.
	\item $\textbf{Transitive: }$ R is transitive if $\forall$ a, b, c $\in$ A, (a, b) $\in$ R and (b, c) $\in$ R implies that (a, c) $\in$ R.
\end{itemize}
For the details about the properties, please check the 4th lecture slide on Moodle. \\

As we solved the problem 3 in PS4 document - which is available on Moodle - in the problem session, we can determine any given relation if it is reflexive, symmetric, anti-symmetric, and transitive.\\

Write an algorithm to determine if a given relation $R$ is reflexive, symmetric, anti-symmetric, and transitive. Your code should meet the following requirements, standards and tasks.

\begin{itemize}
	\item Read the relations in the text file "input.txt".
	\item Let $R$ be a relation on a set $A$ where $\exists a,b \in A, (a,b) \in R$. Each relation $R$ is represented with the lines in the file:
	\begin{itemize}
		\item [1.] The first line says how many relations in $R$.
		\item[2.] The second line gives the elements of the set $A$.
		\item[3.] The following lines give each relation in $R$.
	\end{itemize}
	\item After determining each relation in input.txt whether it is reflexive, symmetric, anti-symmetric and transitive with your algorithm, write its result to the file which is called "output.txt" with the following format.
	\item output.txt:
	\begin{itemize}
		\item[1.] Start a new line with "n" which indicates a new relation.
		\item[2.] The set of $R$
		\item[3.] Reflexive: Yes or No, explain the reason if No (e.g. "(a, a) is not found").
		\item[4.] Symmetric: Yes or No, explain the reason if No (e.g. "(b, a) is not found whereas (a, b) is found.")
		\item[5.] Antisymmetric: Yes or No, explain the reason if No (e.g. "(b, a) and (a, b) are found.")
		\item[6.] Transitive: Yes or No, explain the reason if No (e.g. "(a, c) is not found whereas (a, b) and (b, c) are found.")
	\end{itemize}
	\item An example of the output format is given in "exampleoutput.txt". The file has the result of the first relation in "input.txt".
	\item When explaining why a property does not exist in the relation, one reason is enough to explain if there are more. For example, in "exampleoutput.txt", the relation is not symmetric because (b, a) and (e, a) are not found. Detecting one of them is enough to explain the reason.
	\item $\textbf{Bonus (20 points): }$ If you can explain why a property exists in the relation, it brings you bonus of 20 points.
	\item Your code is responsible to provide exception and error handling. The input file may be given with a wrong information, then your code must be prepared to detect them. For instance, "The element b of the relation (1, b) is not found in the set A = $\{1, 2, 3, 4\}$.".
	\item You can implement your algorithm in Python, Java, C or C++.
	\item $\textbf{Important: }$ Put comments almost for each line of your code to describe what the line is going to do. 
	\item You should put your source code file (file name is problem1.$\{.c, .java, .py, .cpp\}$) and output.txt into your homework zip file (check Course Policy).
\end{itemize}



\end{document} 
